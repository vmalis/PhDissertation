\chapter{Conclusion: Summary of the Developed Methods and Application in Future Studies.}
This chapter is a summary of the results reported in this dissertation and the possible technical and physiology-based extensions of this work in the future. 
%-new paragraph-%

%-new paragraph-%
\textit{Functional MR imaging:} Based on the studies reported in this dissertation a new greatly expanded and improved framework for \textit{in-vivo} evaluation of human muscle dynamics was developed.
The methods presented in this work will allow assessment of detailed 3D strain~/~strain rate patterns in significantly reduced, of up to four times of previous, scan times.
Compressed sensing with established reconstruction and analysis pipeline enables to perform data acquisition and strain~/~strain rate calculations at sub-maximal forces even in senior human subjects.
The physiological findings revealed that shear strain emerged as the primary and significant predictor of both force variation in the aging study and force loss in the disuse atrophy study; the second most important predictor was the normal strain rate in the fiber cross-section. 
The significance of shear strain as an important determinant of force was seen across different analysis: 2D strain rate, 3D strain and strain rate. 
There are several notable observations based on these results: MR measured shear strain is hypothesized to be the shear of the endomysium from adjacent contracting muscle fibers and is the underlying mechanism of lateral transmission of force. 
Validation for this hypothesis comes from computational models by other research groups that explored the transmission pathways for prematurely terminating or damaged muscle fibers. 
The normal strain in the fiber cross-section is constrained by the stiffness of the surrounding matrix (i.e., the endomysium and perimysium). 
Increase in collagen increases the stiffness and the width of endomysium and is known to occur with age and possibly also, to a lesser degree with disuse. 
The changes in the normal strain may reflect the increasing stiffness of the surrounding matrix. 
The last important observation is that the normal strain in the fiber cross-section is highly asymmetric in the young healthy muscle, with deformation occurring almost entirely in the plane of the fibers. 
In disuse atrophy, this anisotropy is reduced significantly. 
Computational models have shown that a deformation constrained to one direction produces a higher force than one without such a constraint. 
This loss in asymmetry of deformation may also partially explain the force loss with unloading. 
The underlying cause for the loss in asymmetry may arise from changes in costameric proteins that link muscle fibers and ECM. 
In summary, the functional studies reported in this dissertation have identified two strain indices as novel predictors of force loss~/~differences. 
Both these strain indices are highly sensitive to the status of the extracellular matrix providing support to conclusions reached in recent animal studies on the role of the ECM in muscle force loss with age. 
However, conclusive proof of the link between MR measured shear strain and lateral transmission of force awaits animal studies where MR and invasive studies can be performed on the same animal. 
Further, imaging based subject specific computational models are required to verify the causal link between shear in the endomysium and the voxel-based shear strain patterns (that can be quantified by MRI).
%-new paragraph-%

%-new paragraph-%
Future studies should include further technical advances towards even more rapid scanning as well as explorations to develop a comprehensive model of muscle force loss. 
In technical work, future work will focus on achieving higher data acquisition acceleration factors integrating more incoherent patterns (e.g., randomization along the velocity dimension) and joint reconstructions and spatial three dimensional compressed sensing acquisition. 
There are also several unanswered questions in physiology that can be addressed using the technical tools developed in the dissertation. 
Immediate future efforts will be focused on validation studies of the strain indices by animal studies and computational modeling. 
Other physiological studies will include extension to disease states such as muscular dystrophy, longitudinal monitoring of exercise regimens and computation of single muscle force.
%-new paragraph-%

%-new paragraph-%
\textit{Diffusion tensor imaging and modeling:} The study presented in chapter~\ref{ch: DiffusionExp} has demonstrated that diffusion models can relate changes observed at the voxel level to changes in the tissue microarchitecture
The two-compartment model applied to DTI data pre- and post-suspension showed that muscle fiber, permeability and intracellular volume decreased and collagen increased while muscle ellipticity decreased post-suspension; the latter could potentially be explained by the lack of a mechanical stimulus during unloading. 
The RPBM was applied to time dependent DTI data on old and young cohorts.
Model derived volume fraction (measure of the membrane's ability to hinder diffusion) decreased with age while diffusion time and residence time in a cell increased with age while other model parameters such as the free diffusion coefficient, the fiber size, and membrane permeability did not.
Modeling can be further improved by limiting the range for the parameters used in the fit.
This can be achieved by acquiring additional data with another imaging techniques such as Ultra-low TE (TE) and quantitative Magnetization Transfer (qMT).
Diffusion model-derived parameters can be used to generate subject specific computational models allowing predictions of force and strain distributions; the models in turn can be validated using the functional MR data.
%-new paragraph-%

%-new paragraph-%
Future work in diffusion tensor imaging will focus on technical developments including sequence development (e.g., Q-ball imaging, multi-shell imaging) as well as extensions to the currently proposed muscle diffusion models. 
On-going work in our group includes validation of the fiber diameters and permeability from the DTI model to corresponding values determined from biopsy of muscle tissue on the same subject. 
The reference values from the biopsy analysis will be used to fine tune the DTI model. 
The physiological applications of DTI modeling extend from monitoring structural changes in normal aging, longitudinal monitoring of interventions to tracking disease progress.
%-new paragraph-%

%-new paragraph-%
The functional and structural MR imaging reported in this dissertation are part of a bigger goal in Prof. Sinha's group: to develop MR techniques for comprehensive assessment of muscle with a focus on the extracellular matrix.
 These imaging biomarkers, once established, will be able to advance the diagnosis, tracking of natural progression, and response to therapy in different musculoskeletal conditions ranging from sarcopenia, disuse atrophy to dystrophies. 