\chapter{Conclusion: Summary of the Developed Methods and Application in Future Studies.}
Advanced MRI techniques provide a number of indices to probe muscle fiber and the extracellular matrix. 

\textit{Functional MR imaging:} Based on the earlier conducted studies a new greatly expanded and improved framework for \textit{in-vivo} evaluation of human muscle dynamics was developed.
The methods presented in this work allow assessment of detailed 3d strain/strain rate patterns with up to 4 times reduced scan time.
Compressed sensing with established reconstruction and analysis pipeline allow to perform data acquisition and strain/strain rate calculations at submaximal forces even in senior human subjects.
Application of the framework demonstrates that strain and strain rate are significant predictors of force variability with age and disuse atrophy.
Future studies may achieve even higher data acquisition acceleration factors by introducing more incoherent patterns (randomization along the velocity dimension).
In addition analysis of the fiber aligned strain/strain rate is yet at the very preliminary stage and may provide more insights.
%-new paragraph-%

%-new paragraph-%
\textit{Diffusion tensor imaging and modeling:} The study presented in chapter~\ref{ch: DiffusionExp} has demonstrated that diffusion models can relate changes observed at the voxel level to changes in the tissue microarchitecture
The two-compartment model applied to DTI data pre- and post-suspension showed that muscle fiber, permeability and intracellular volume decreased and collagen increased while muscle ellipticity decreased post-suspension; the latter could potentially be explained by the lack of a mechanical stimulus during unloading. 
The RPBM was applied to time dependent DTI data on old and young cohorts.
Model derived volume fraction (measure of the membrane’s ability to hinder diffusion) decreased with age while diffusion time and residence time in a cell increased with age while other model parameters such as the free diffusion coefficient, the fiber size, and membrane permeability did not.
Modeling can be further improved by limiting the range for the parameters used in the fit.
This can be achieved by acquiring additional data with another imaging techniques such as Ultra-low TE (TE) and quantitative Magnetization Transfer (qMT).
Diffusion model-derived parameters can be used to generate subject specific computational models allowing predictions of force and strain distributions; the models in turn can be validated using the functional MR data.
%-new paragraph-%

%-new paragraph-%
Future simulation studies using computational models may provide the causal link between shear in the ECM, shear strain from MRI and lateral force transmission pathways allowing one to establish shear strain from MRI as a surrogate marker of LTF. 
In addition, a direct way to establish the link between \textit{in-vivo} MR strain indices and LTF will require animal model studies. 
MR imaging biomarkers, once established, will be able to advance the diagnosis, tracking of natural progression and response to therapy in different musculoskeletal conditions ranging from sarcopenia, disuse atrophy to dystrophies.