%
%
% UCSD Doctoral Dissertation Template
% -----------------------------------
% http://ucsd-thesis.googlecode.com
%
%


%% REQUIRED FIELDS -- Replace with the values appropriate to you

% No symbols, formulas, superscripts, or Greek letters are allowed
% in your title.
\title{Study of Human Muscle Structure and Function with Velocity Encoded Phase Contrast and Diffusion Tensor Magnetic Resonance Imaging Techniques}

\author{Vadim Malis}
\degreeyear{\the\year}

% Master's Degree theses will NOT be formatted properly with this file.
\degreetitle{Doctor of Philosophy}

\field{Physics}
%\specialization{Anthropogeny}  % If you have a specialization, add it here

\chair{Professor Shantanu Sinha}
% Uncomment the next line iff you have a Co-Chair
\cochair{Professor Henry Abarbanel}
%
% Or, uncomment the next line iff you have two equal Co-Chairs.
%\cochairs{Professor Chair Masterish}{Professor Chair Masterish}

%  The rest of the committee members  must be alphabetized by last name.
\othermembers{
Professor Jiang Du\\
Professor Alexander Groisman\\
Professor Elena Koslover\\
Professor Usha Sinha
}
\numberofmembers{6} % |chair| + |cochair| + |othermembers|


%% START THE FRONTMATTER
%
\begin{frontmatter}

%% TITLE PAGES
%
%  This command generates the title, copyright, and signature pages.
%
\makefrontmatter

%% DEDICATION
%
%  You have three choices here:
%    1. Use the ``dedication'' environment.
%       Put in the text you want, and everything will be formated for
%       you. You'll get a perfectly respectable dedication page.
%
%
%    2. Use the ``mydedication'' environment.  If you don't like the
%       formatting of option 1, use this environment and format things
%       however you wish.
%
%    3. If you don't want a dedication, it's not required.
%
%
\begin{dedication}
\end{dedication}


% \begin{mydedication} % You are responsible for formatting here.
%   \vspace{1in}
%   \begin{flushleft}
% 	To me.
%   \end{flushleft}
%
%   \vspace{2in}
%   \begin{center}
% 	And you.
%   \end{center}
%
%   \vspace{2in}
%   \begin{flushright}
% 	Which equals us.
%   \end{flushright}
% \end{mydedication}



%% EPIGRAPH
%
%  The same choices that applied to the dedication apply here.
%
\begin{epigraph} % The style file will position the text for you.
  \emph{It was eerie. I saw myself in that machine.\\
  I never thought my work would come to this. }\\
  --- Isidor Isaac Rabi
\end{epigraph}

% \begin{myepigraph} % You position the text yourself.
%   \vfil
%   \begin{center}
%     {\bf Think! It ain't illegal yet.}
%
% 	\emph{---George Clinton}
%   \end{center}
% \end{myepigraph}


%% SETUP THE TABLE OF CONTENTS
%
\tableofcontents
\listoffigures  % Comment if you don't have any figures
\listoftables   % Comment if you don't have any tables



%% ACKNOWLEDGEMENTS
%
%  While technically optional, you probably have someone to thank.
%  Also, a paragraph acknowledging all coauthors and publishers (if
%  you have any) is required in the acknowledgements page and as the
%  last paragraph of text at the end of each respective chapter. See
%  the OGS Formatting Manual for more information.
%
\begin{acknowledgements}

Section~\ref{sec: SR_ULLS} is a reprint of material, with minor edits as it appears in: V.~Malis, U.~Sinha, R.~Csapo, M.~Narici, and S.~Sinha, ``Relationship of changes in strain rate indices estimated from velocity-encoded MR imaging to loss of muscle force following disuse atrophy,'' \emph{Magn. Reson. Med.}, vol. 79, no. 2, pp. 912-922, Feb. 2018.
%-new paragraph-%

%-new paragraph-%
Section~\ref{sec: SR_SHEAR} is a reprint of material, with minor edits as it appears in: U.~Sinha, V.~Malis, R.~Csapo, M.~Narici, and S.~Sinha, ``Shear strain rate from phase contrast velocity encoded MRI: Application to study effects of aging in the medial gastrocnemius muscle,'' \emph{J. Magn. Reson. Imaging}, vol. 48, no. 5, pp. 1351-1357, Nov. 2018.
%-new paragraph-%

%-new paragraph-%
Section~\ref{sec: CS_paper} is a reprint of material, with minor edits as it appears in: V.~Malis, U.~Sinha, and S.~Sinha, ``Compressed sensing velocity encoded phase contrast imaging: Monitoring skeletal muscle kinematics,'' \emph{Magn. Reson. Med.}, Dec. 2019.
%-new paragraph-%

%-new paragraph-%
Section~\ref{sec: CS_SRYO} is a reprint of material, with minor edits as it appears in: V.~Malis, U.~Sinha, and S.~Sinha, ``Principal Axis and Fiber Aligned 3D Strain / Strain Rate Mapping with Compressed Sensing Velocity Encoded Phase Contrast MRI to study Aging Muscle,'' \emph{Proceedings of the International Society of Magnetic Resonance in Medicine}, Sydney, 2020.
%-new paragraph-%

%-new paragraph-%
Section~\ref{sec: DTI ULLS} is a reprint of material, with additional details provided in subsection~\ref{subsec: bicompart}, as it appears in: V.~Malis, U.~Sinha, R.~Csapo, M.~Narici, E.~Smitaman, and S.~Sinha, ``Diffusion tensor imaging and diffusion modeling: Application to monitoring changes in the medial gastrocnemius in disuse atrophy induced by unilateral limb suspension,'' \emph{J. Magn. Reson. Imaging}, vol. 49, no. 6, pp. 1655-1664, Dec. 2018.
%-new paragraph-%

%-new paragraph-%
Section~\ref{sec: STEAM RPBM} is a reprint of material, with additional details, as it appears in: V.~Malis, S.~Sinha, E.~Smitaman, and U.~Sinha, ``Skeletal Muscle Diffusion Modeling to Identify Age Related Remodeling of Muscle Microstructure,'' \emph{Proceedings of the International Society of Magnetic Resonance in Medicine}, Sydney, 2020.
%-new paragraph-%

%-new paragraph-%
The author of the dissertation was the primary author of these papers and abstracts.
\end{acknowledgements}


%% VITA
%
%  A brief vita is required in a doctoral thesis. See the OGS
%  Formatting Manual for more information.
%
\begin{vitapage}
\begin{vita}
  \item[2020] Ph.~D. in Physics, University of California San Diego
  \item[2014] M.~S. in Physics, San Diego State University
  \item[2011] B.~S. in Physics, Moscow State Lomonosov University, Russia
  \end{vita}
\begin{publications}
  \item   V. Malis, U. Sinha, and S. Sinha, ``Compressed sensing velocity encoded phase contrast imaging: Monitoring skeletal muscle kinematics,'' \emph{Magn. Reson. Med.}, Dec. 2019.
  \item V. Malis, U. Sinha, R. Csapo, M. Narici, E. Smitaman, and S. Sinha, ``Diffusion tensor imaging and diffusion modeling: Application to monitoring changes in the medial gastrocnemius in disuse atrophy induced by unilateral limb suspension,'' \emph{J. Magn. Reson. Imaging}, vol. 49, no. 6, pp. 1655-1664, Dec. 2018.
  \item U. Sinha, V. Malis, R. Csapo, M. Narici, and S. Sinha, ``Shear strain rate from phase contrast velocity encoded MRI: Application to study effects of aging in the medial gastrocnemius muscle,'' \emph{J. Magn. Reson. Imaging}, vol. 48, no. 5, pp. 1351-1357, Nov. 2018.
  \item	V. Malis, U. Sinha, R. Csapo, M. Narici, and S. Sinha, ``Relationship of changes in strain rate indices estimated from velocity-encoded MR imaging to loss of muscle force following disuse atrophy,'' \emph{Magn. Reson. Med.}, vol. 79, no. 2, pp. 912-922, Feb. 2018.
  \item	V. Ugarte, U. Sinha, V. Malis, R. Csapo, and S. Sinha, ``3D multimodal spatial fuzzy segmentation of intramuscular connective and adipose tissue from ultrashort TE MR images of calf muscle,'' \emph{Magn. Reson. Med.}, vol. 77, no. 2, pp. 870-883, Feb. 2017.
  \item	J.-S. Chen, R. R. Basava, Y. Zhang, R. Csapo, V. Malis, U. Sinha, J. Hodgson, and S. Sinha,``Pixel-based meshfree modelling of skeletal muscles,'' \emph{Comput Methods Biomech Biomed Eng Imaging Vis}, vol. 4, no. 2, pp. 73-85, 2016.
  \item	R. Csapo, V. Malis, U. Sinha, and S. Sinha, ``Mapping of spatial and temporal heterogeneity of plantar flexor muscle activity during isometric contraction - correlation of Velocity-Encoded MRI with EMG,'' \emph{J. Appl. Physiol.}, vol. 119, no. 5, pp. 558-568, Sep. 2015.
  \item U. Sinha, V. Malis, R. Csapo, A. Moghadasi, R. Kinugasa, and S. Sinha, ``Age-related differences in strain rate tensor of the medial gastrocnemius muscle during passive plantarflexion and active isometric contraction using velocity encoded MR imaging: potential index of lateral force transmission,'' \emph{Magn. Reson. Med.}, vol. 73, no. 5, pp. 1852-1863, May 2015.
  \item U. Sinha, R. Csapo, V. Malis, Y. Xue, and S. Sinha, ``Age-related differences in diffusion tensor indices and fiber architecture in the medial and lateral gastrocnemius,'' \emph{J. Magn. Reson. Imaging}, vol. 41, no. 4, pp. 941-953, Apr. 2015.
  \item R. Csapo, V. Malis, U. Sinha, J. Du, and S. Sinha, ``Age-associated differences in triceps surae muscle composition and strength - an MRI-based cross-sectional comparison of contractile, adipose and connective tissue,'' \emph{BMC Musculoskelet Disord}, vol. 15, no. 1, p. 209, Jun. 2014.
  \item R. Csapo, V. Malis, J. Hodgson, and S. Sinha, ``Age-related greater Achilles tendon compliance is not associated with larger plantar flexor muscle fascicle strains in senior women,'' \emph{J. Appl. Physiol.}, vol. 116, no. 8, pp. 961-969, Apr. 2014.
  %\item Your Name, ``A Simple Proof Of The Riemann Hypothesis'', \emph{Annals of Math}, 314, 2007.
\end{publications}

\begin{talks}	
	\item V. Malis, U. Sinha, and S. Sinha, ``Principal Axis and Fiber Aligned 3D Strain / Strain Rate Mapping with Compressed Sensing Velocity Encoded Phase Contrast MRI to study Aging Muscle,'' \emph{Proceedings of the International Society of Magnetic Resonance in Medicine}, Sydney, 2020
	\item V. Malis, S. Sinha, E. Smitaman, and U. Sinha, ``Skeletal Muscle Diffusion Modeling to Identify Age Related Remodeling of Muscle Microstructure,'' \emph{Proceedings of the International Society of Magnetic Resonance in Medicine}, Sydney, 2020
	\item U. Sinha , V. Malis , E. Smitaman, and S. Sinha, ``Short T2 Fraction Mapping of Skeletal Muscle Integrating Ultralow TE (UTE) and MP- IDEAL multiecho data,'' \emph{Proceedings of the International Society of Magnetic Resonance in Medicine}, Sydney, 2020
	\item S. Sinha, V. Malis, and U. Sinha, ``Mapping of Muscle Strain Rate at Varying Force Levels of Isometric Contraction, with Compressed-Sensing Velocity-Encoded Phase-Contrast MR Imaging,'' \emph{Proceedings of the International Society of Magnetic Resonance in Medicine}, Montreal, 2019
	\item V. Malis, U. Sinha, and S. Sinha, Compressed Sensing Velocity Encoded Phase Contrast Imaging: Monitoring Skeletal Muscle kinematics, \emph{Proceedings of the International Society of Magnetic Resonance in Medicine}, Montreal, 2019
	\item V. Malis, U. Sinha, and S. Sinha, ``Permeable Barrier Modeling of Age Induced Changes in the Time Dependent Diffusion Eigenvalues,'' \emph{Proceedings of the International Society of Magnetic Resonance in Medicine}, Montreal, 2019
	\item V. Malis, U. Sinha, and S. Sinha, ``Changes in strain tensor resulting from atrophy induced by Unilateral Limb Suspension of the calf muscle,'' \emph{Proceedings of the International Society of Magnetic Resonance in Medicine}, Montreal, 2019
	\item U. Sinha, V. Malis, and S. Sinha, ``3D Fiber Aligned Strain Rate: Application to Unilateral Limb Suspension Induced Atrophy,'' \emph{Proceedings of the International Society of Magnetic Resonance in Medicine}, Paris, 2018
	\item S. Sinha, U. Sinha, and V. Malis, ``Variation of Strain Rate with Force Output in the Medial Gastrocnemius During Isometric Contractions in Young and Senior Subjects,'' \emph{Proceedings of the International Society of Magnetic Resonance in Medicine}, Paris, 2018
	\item S. Sinha, U. Sinha, and V. Malis, ``Variation of Strain Rate with Force Output in the Medial Gastrocnemius During Isometric Contractions in Young and Senior Subjects,'' \emph{Proceedings of the International Society of Magnetic Resonance in Medicine}, Paris, 2018	
	\item T. Oda, S. Sinha, and V. Malis, ``Heterogeneity of Quadriceps Muscle Activation during Isometric Contractions as revealed by Velocity Encoded Phase Contrast (VE-PC) Image Mapping,'' \emph{Proceedings of the International Society of Magnetic Resonance in Medicine}, Honolulu, HI, 2017	
	\item V. Ugarte, U. Sinha, V. Malis, and S. Sinha, ``Automated segmentation of Intramuscular Connective Tissue (IMCT) from skeletal muscle in presence of artifacts: Application to Changes in IMCT in an Unilateral Limb Suspension Induced Acute Atrophy Model in the Plantarflexors,'' \emph{Proceedings of the International Society of Magnetic Resonance in Medicine}, Honolulu, HI, 2017
	\item V. Malis, U. Sinha, R. Csapo, and S. Sinha, ``Age Related Differences in Shear Strain in Medial Gastrocnemius: Implications for Lateral Transmission of Force,'' \emph{Proceedings of the International Society of Magnetic Resonance in Medicine}, Honolulu, HI, 2017
	\item U. Sinha, V. Malis, and S. Sinha, ``Two compartmental diffusion model of skeletal muscle: application to aging and chronic limb suspension induced DTI changes in the medial gastrocnemius,'' \emph{Proceedings of the International Society of Magnetic Resonance in Medicine}, Honolulu, HI, 2017
	\item V. Malis, U. Sinha, R. Csapo, and S. Sinha, ``Functional Changes in Medial Gastrocnemius from Unilateral Limb Suspension Induced Acute Atrophy: a 2D Strain Rate Study during Isometric Contraction,'' \emph{Proceedings of the International Society of Magnetic Resonance in Medicine}, Singapore, 2016
	\item U. Sinha, V. Malis, R. Csapo, and S. Sinha, ``Physiological insights into medial gastrocnemius function during eccentric contraction in normal and in acute atrophy - Quantification of 2D strain rate indices from Velocity Encoded Phase Contrast MR Imaging,'' \emph{Proceedings of the International Society of Magnetic Resonance in Medicine}, Singapore, 2016
	\item S. Sinha, V. Malis, R. Csapo, J. Du, and U. Sinha, ``Morphological, Compositional, Fiber Architectural Changes in from Unilateral Limb Suspension Induced Acute Atrophy Model in the Medial Gastrocnemius Muscle,'' \emph{Proceedings of the International Society of Magnetic Resonance in Medicine}, Singapore, 2016
	\item Y. Xue, U. Sinha, V. Malis, R. Csapo, and S. Sinha, ``3D Curvature of Medial Gastrocnemius (MG) Muscle Fibers Tracked from Diffusion Tensor Images (DTI): Age Related Differences in 3D fiber curvature,'' \emph{Proceedings of the International Society of Magnetic Resonance in Medicine}, Singapore, 2016
	\item V. Malis, U. Sinha, R. Csapo, and S. Sinha, ``3D Shear strain analysis of Medial Gastrocnemius muscle based on Velocity Encoded and Diffusion Tensor Imaging data,'' \emph{Proceedings of the International Society of Magnetic Resonance in Medicine}, Milan, 2014
	\item R. Csapo, V. Malis, U. Sinha, J. Du, and S. Sinha, ``Age-associated Changes in Triceps Surae Muscle Composition and Plantarflexor Strength - an MR imaging based Study with Ultra-short Echo-time (UTE) and Fat-Water Quantification of Connective, Adipose and Contractile Tissues,'' \emph{Proceedings of the International Society of Magnetic Resonance in Medicine}, Milan, 2014
	\item V.Ugarte, V. Malis, U. Sinha, R. Csapo, and S. Sinha, ``3D Multimodal spatial fuzzy segmentation of intramuscular connective and adipose tissue from ultralow TE MR,'' \emph{Proceedings of the International Society of Magnetic Resonance in Medicine}, Milan, 2014
	\item R. Csapo, V. Malis, J. Hodgson, and S. Sinha, ``Phase-contrast MR imaging reveals age-associated differences in plantarflexor fascicle and aponeurosis behavior in isometric contractions,'' \emph{Proceedings of the International Society of Magnetic Resonance in Medicine}, Milan, 2014
	\item Yanjie Xue, U. Sinha, V. Malis, R. Csapo, and S. Sinha, ``Changes in the Medial and Lateral Gastrocnemius Fiber Architecture with Age,'' \emph{Proceedings of the International Society of Magnetic Resonance in Medicine}, Milan, 2014
	\item U. Sinha, R. Csapo, V. Malis, Y. Xue, and S. Sinha,``Age Related Changes in Diffusion Tensor Indices in the Medial and Lateral Gastrocnemius,'' \emph{Proceedings of the International Society of Magnetic Resonance in Medicine}, Milan, 2014

\end{talks}

\begin{awards}
	\item 2017 ISMRM Summa Cum Laude	
\end{awards}


\end{vitapage}


%% ABSTRACT
%
%  Doctoral dissertation abstracts should not exceed 350 words.
%   The abstract may continue to a second page if necessary.
%
\begin{abstract}
The disproportionate loss of muscle force with aging and disuse atrophy compared to the loss of muscle mass is not yet completely understood. 
In addition to well-established neural and contractile determinants of force loss,  remodeling of the extracellular matrix (ECM) has been recently shown in animal models to be another important contributor.  
\textit{In-vivo} human studies exploring the structural remodeling of the ECM and its functional consequences are lacking due to the paucity of appropriate  imaging techniques.  
This study focuses on the development and application of advanced Magnetic Resonance Imaging (MRI) methods to elucidate the mechanisms of loss of force with aging and disuse atrophy with the focus on ECM.
%-new paragraph-%

%-new paragraph-%
Functional changes are investigated by strain and strain rate tensor mapping of muscle under different contraction paradigms using Velocity Encoded Phase-Contrast MRI. 
Methodological advances include improvements in hardware and software control of the dynamic studies.  
To overcome the limitation of long scan times, compressed sensing MR acquisition and reconstruction framework to reduce scan times to under a minute were developed.  
A multi-step automated analysis pipeline to extract 3D strain / strain rate tensors from the velocity images was implemented to process the large dynamic volumes.  
Strain indices reflecting the material properties of the ECM were shown to correlate with force loss leading to a hypothesis that shear strain may serve as a surrogate marker for lateral transmission of force.
%-new paragraph-%

%-new paragraph-% 
Diffusion tensor imaging has been applied to study skeletal muscle fiber architecture. 
Since the resolution of the images precludes direct inferences about the microstructure, we applied bicompartmental and Random Permeable Barrier models of diffusion to the diffusion data obtained with spin-echo and custom-developed stimulated echo echo-planar-imaging sequences respectively. 
Model derived parameters obtained from fitting time-dependent diffusion data were in physiologically reasonable range, with potential for tracking age-related changes in muscle microstructure. 
%-new paragraph-%

%-new paragraph-%
The developed  imaging and modeling techniques were applied to a cohort of young/ senior subjects and to longitudinal tracking of disuse atrophy induced by Unilateral Limb Suspension.  
These studies may potentially provide the causal link between age- and disuse-related structural remodeling and its functional consequences. 
\end{abstract}


\end{frontmatter}
