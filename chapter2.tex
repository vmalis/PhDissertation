%#########################################################
\chapter{Fundamentals of MRI}
%#########################################################
Among various medical imaging modalities Magnetic Resonance Imaging (MRI) is currently the most powerful technique to study structure and functions of human body \textit{in-vivo} and non-invasively at the level of the detail that is not available with any other imaging method. 
The power of MRI comes from the ability to to obtain quantitative spatial information for different types of human body tissue and fluids placed inside a strong magnetic field by manipulating Radio-frequency (RF) and gradient pulses. 
Magnetic Resonance Imaging is a fascinating combination of quantum physics, electrodynamics, advanced engineering and computation methods. 
This chapter is a short introduction to the fundamentals of MRI.
%=========================================================
\section{Larmor Precession}
%=========================================================
A charged particle placed in an external magnetic field will precess at a frequency proportional to this external field. 
In classical description magnetic dipole moment for a charged particle $\boldsymbol\mu$ is proportional to the angular momentum $\mathbf{J}$:
%.........................................................
\begin{equation}\label{eq: Classical magnetic moment}
	\boldsymbol{\mu}=\frac{q}{2m}\mathbf{J}
\end{equation}
%.........................................................
where $q$ is the charge of a particle and $m$ is its mass. 
Torque on a magnetic dipole in external magnetic filed $\mathbf{B}$ is:
%.........................................................
\begin{equation}\label{eq: Torque}
	\boldsymbol{\tau}=\frac{d\mathbf{J}}{dt}=\boldsymbol{\mu}\times\mathbf{B}=\frac{q}{2m}\mathbf{J}\times\mathbf{B}
\end{equation}
%.........................................................
Precessional frequency is then:
%.........................................................
\begin{equation}\label{eq: Classic Larmor frequency}
	\omega=\frac{d\phi}{dt}=-\frac{d\phi}{d\mathbf{J}}\cdot \frac{d\mathbf{J}}{dt}=\frac{1}{J\sin{\theta}}\cdot \frac{q}{2m}\ - J B \sin{\theta}
\end{equation}
%.........................................................
where $d\phi$ is the rotation angle and $\theta$ is the angle enclosed by angular momentum $\mathbf{J}$ and magnetic field $\mathbf{B}$:
%.........................................................
\begin{equation}\label{eq: Classic Larmor frequency 2}
	\omega=-\frac{q}{2m}\ B
\end{equation}
%.........................................................
In quantum mechanics description magnetic dipole moment is quantized:
%.........................................................
\begin{equation}\label{eq: Quantum magnetic moment}
	\boldsymbol{\mu}=\gamma\hbar \mathbf{I}
\end{equation}
%.........................................................
where  $\gamma$ is gyromagnetic ratio (for protons: $\gamma/2\pi = \SI{42.57}{\mega\hertz /\tesla}$), $\hbar = \SI{1.05e-34}{\joule \cdot \second}$ is reduced Planck constant and $\mathbf{I}$ is nuclear spin angular momentum. For each proton energy is:
%.........................................................
\begin{equation}\label{eq: Quantum energy}
	\varepsilon=-\boldsymbol{\mu}\cdot \mathbf{B}=-\gamma\hbar\mathbf{I}\cdot\mathbf{B}
\end{equation}
%.........................................................
and since there are only two possible states (magnetic moment is almost parallel or anti-parallel to the external magnetic field $\mathbf{B}$) with $ I = \pm 1/2$  the energy difference between these states is:
%.........................................................
\begin{equation}\label{eq: Quantum energy delta}
	\Delta\varepsilon=-\left(\frac{1}{2} + \frac{1}{2}\right)\gamma\hbar B
\end{equation}
%.........................................................
using Einstein-Planck relation $\Delta\varepsilon=\hbar\omega$ and Equation~\ref{eq: Quantum energy delta} the Larmor frequency:
%.........................................................
\begin{equation}\label{eq: Quantum Larmor frequency}
	\omega=-\gamma B
\end{equation} 
%.........................................................
Comparing Equations~\ref{eq: Classic Larmor frequency 2} and \ref{eq: Quantum Larmor frequency} one can see that the precessional frequency is same as the frequency of the photon that can be absorbed or emitted by proton to go from one energy state to another.
%=========================================================
\section{Population of Energy States}
%=========================================================
The signal measured in MRI is the net magnetization $\mathbf{M_{0}}$ which is a result of difference in proton energy level population density given by Boltzmann distribution:
%.........................................................
\begin{equation}\label{eq: Boltzmann distribution}
	\frac{N_{\mathrm{up}}}{N_{\mathrm{down}}}=\exp\left(\frac{-\Delta\varepsilon}{k_{B}T}\right)\approx 1 + \frac{{\gamma\hbar B}}{k_{B}T}
\end{equation}
%.........................................................
where $k_{B} = \SI{1.38e-23}{\joule \cdot {\kelvin}^{-1}}$ is Boltzmann constant the difference is then:
%.........................................................
\begin{equation} \label{eq: Population difference}
	\Delta N = \frac{N_{\mathrm{total}}}{2}\frac{\gamma\hbar B}{k_{B}T}
\end{equation}
%.........................................................
multiplying Equation~\ref{eq: Population difference} by magnetic moment $\mu = \gamma \hbar / 2$ and dividing by a total number of protons gives the equation for magnetization per unit volume:
%.........................................................
\begin{equation}\label{eq: Magnetization per volume}
	\mathbf{M_{0}}=\frac{\rho\gamma^2\hbar^2\mathbf{B}}{4k_{B}T}
\end{equation}
%.........................................................
where $\rho$ is proton density. 
From the Equation~\ref{eq: Magnetization per volume} one can see that the measured signal is directly proportional to the strength of the external magnetic field and inversely proportional to the tissue or fluid temperature. 
The direction of net magnetization is matched with the direction of the external magnetic field $\mathbf{B}$ since there are more protons in a low level energy state (aligned with external magnetic field). 
Equation~\ref{eq: Magnetization per volume} provides an estimate for the signal to be measured. 
Simple example would be magnetization for $\SI{1}{\milli\liter}$ of water at temperature $\SI{310}{\kelvin}$) placed in the $\SI{1}{\tesla}$ magnetic field. 
Calculating proton density $\rho$ of water using the Avogadro number gives:
%.........................................................
\begin{equation}\label{eq: Magnetization estimation}
	\mathbf{M_{0}}\approx \SI{3}{\milli\ampere / \meter} 
\end{equation}
%.........................................................
Compared to the strength of the external magnetic field $\mathbf{B}$ the magnetization of the tissue of interest is much smaller which makes it hard to measure when at equilibrium. 
If tipped to the transverse plane by applying RF-pulse the signal becomes distinct and easier to measure.
%=========================================================
\section{Bloch Equations}
%=========================================================
The evolution of net magnetization is described by Bloch equation~\cite{Bloch1946}:
%.........................................................
\begin{equation}\label{eq: Bloch equation}
	\frac{d\mathbf{M}}{dt}=\gamma\mathbf{M} \times \mathbf{B}
\end{equation}
%.........................................................
relaxation and diffusion terms are dropped to simplify further examination. To flip magnetization to the plane perpendicular to the direction of the applied strong magnetic field and produce transverse magnetization an RF-pulse should be applied. A common approach is to switch from laboratory reference frame defined with respect to the scanner to the frame rotating at some constant angular frequency $\Omega$. Magnetic field $\mathbf{B}$ in the laboratory reference frame can be chosen as following form:
%.........................................................
\begin{equation}\label{eq: External B field}
	\mathbf{B}=\hat{x}B_{1}(t)\cos{\omega_{\text{rf}} t}-\hat{y}B_{1}(t)\sin{\omega_{\text{rf}} t} +\hat{z}B_0
\end{equation}
%.........................................................
where $B_0$ is constant strong magnetic field aligned with the $z$-axis and $B_1$ is the time-varying component in the transverse plane. 
Rewriting Bloch equation using Equation~\ref{eq: External B field}:
%.........................................................
\begin{equation}\label{eq: Bloch equation lab frame}
\left(\frac{d\mathbf{M}}{dt}\right)_{\text{lab}}=\gamma\mathbf{M}\times\biggl[\hat{x}B_1(t)\cos{\omega_{\text{rf}}t}-\hat{y}\sin{\omega_{\text{rf}}t}+\hat{z}B_0\biggr]
\end{equation}\\
%.........................................................
In matrix form and frame of reference rotating at constant frequency $\Omega$ time-varying field $\mathbf{B_{1}}$ is the following:
%.........................................................
\begin{equation}\label{eq: External B field rotating frame}
\begin{bmatrix}
    B_{1,x}(t)\\
    B_{1,y}(t)\\
    B_{1,z}(t)\\
\end{bmatrix}_{\mathrm{rot}} = 
	\begin{bmatrix}
    \cos{\Omega t} & -\sin{\Omega t} & 0\\
    \sin{\Omega t} & \cos{\Omega t} & 0\\
    0 & 0 & 1\\
	\end{bmatrix}
	\begin{bmatrix}
    \phantom{-}B_{1}(t)\cos{\omega_{\text{rf}} t}\\
    -B_{1}(t)\sin{\omega_{\text{rf}} t}\\
    0\\
	\end{bmatrix}
\end{equation}
%.........................................................
after performing multiplication and making use of trigonometric identities the transverse component $\mathbf{B_{1}}$ in the rotating frame of reference is:
%.........................................................
\begin{equation}\label{eq: External B field rotating frame 2}
	\mathbf{B_{1}}_{\mathrm{rot}}(t)=
	\begin{bmatrix}
	B_{1}(t) \cos(\Omega -\omega_{\text{rf}})t\\
	B_{1}(t) \sin(\Omega -\omega_{\text{rf}})t\\
	0	
	\end{bmatrix}
\end{equation}
%.........................................................
Closely following~\cite{Slichter1990, RNDT24} Equation~\ref{eq: Bloch equation lab frame} the Bloch equation in the rotating frame becomes: 
%.........................................................
\begin{equation}\label{eq: Bloch equation rot}
\begin{split}
	\left(\frac{d\mathbf{M}}{dt}\right)_{\text{rot}} &=\left(\frac{d\mathbf{M}}{dt}\right)_{\text{lab}} + \Omega \hat{z} \times \mathbf{M} =\\ \\
	&= \gamma\mathbf{M}\times\biggl[B_1(t)(\hat{x}\cos{(\omega_{\text{rf}}-\Omega)}t-\hat{y}\sin{(\omega_{\text{rf}}-\Omega)t)}+
		 \hat{z}\left(B_0-\frac{\Omega}{\gamma}\right)\biggr]
\end{split}	
\end{equation}\\ \\
%.........................................................
Carrying out vector cross product the system of equation in scalar form:
%.........................................................
\begin{equation}\label{eq: scalar form of Bloch}
\begin{aligned}
	\left(\frac{M_x}{dt}\right)_\mathrm{rot} &= \gamma M_y\left(B_0-\frac{\Omega}{\gamma}\right)+\gamma M_zB_1(t)\sin(\omega_{\text{rf}}-\Omega)t\\
	\left(\frac{M_y}{dt}\right)_\mathrm{rot} &= -\gamma M_x\left(B_0-\frac{\Omega}{\gamma}\right)+\gamma M_zB_1(t)\cos(\omega_{\text{rf}}-\Omega)t\\
	\left(\frac{M_z}{dt}\right)_\mathrm{rot} &= -\gamma M_xB_1(t)\sin(\omega_{\text{rf}}-\Omega)t - \gamma M_yB_1(t)\cos(\omega_{\text{rf}}-\Omega)t
\end{aligned}
\end{equation}
%.........................................................
Three special cases are:
\begin{itemize}
	\item \textit{RF reference frame: }$\Omega=\omega_{\text{rf}}$\\
	Frequency of rotating frame $\Omega$ same as the RF-pulse frequency $\omega_{\mathrm{rf}}$
%.........................................................
\begin{equation}\label{eq: Bloch case 1}
	\left(\frac{d\mathbf{M}}{dt}\right)_{\text{rot}}=\gamma\mathbf{M}\times\left[\hat{x}B_1(t)+\hat{z}\left(B_0-\frac{\omega_{\text{rf}}}{\gamma}\right)\right]	
\end{equation}
%.........................................................
here $B_1$ field is stationary. and due to the choice for $\mathbf{B}$ in Equation~\ref{eq: External B field} the filed $B_1$ is oriented along $x$ axis. 
In general $B_1$ field can have both $x$ and $y$ component.
\item \textit{Larmor reference frame: }$\Omega=\omega$\\
Rotating frame frequency $\Omega$ equals Larmor frequency $\omega = \gamma B_{0}$
%.........................................................
\begin{equation}\label{eq: Bloch case 2}
	\left(\frac{d\mathbf{M}}{dt}\right)_{\text{rot}}=\gamma\mathbf{M}\times\left[B_1(t)(\hat{x}\cos{(\omega_{\text{rf}}-\omega)}t-\hat{y}\sin{(\omega_{\text{rf}}-\omega)t)}\right]
\end{equation}
%.........................................................
in this case the am$B_0$ filed is eliminated.
\item \textit{Resonance case:} $\omega=\omega_{\text{rf}}=\omega_{0}$
%.........................................................
\begin{equation}\label{eq: Bloch case 3}
	\left(\frac{d\mathbf{M}}{dt}\right)_{\text{rot}}=\gamma\mathbf{M}\times \hat{x}B_1(t)
\end{equation}
%.........................................................
\end{itemize}
Rotating reference frame vastly simplifies the study of the time evolution of net magnetization by transforming rapidly oscillating RF field into the time-dependent field $B_1(t)$. 
Since $x$ and $y$ components of magnetization is of the most interest a complex magnetization $\bar{M}$ is usually introduced:
%.........................................................
\begin{equation}\label{eq: Transverse magnetization}
	\bar{M}=M_x+iM_y
\end{equation}
%.........................................................
which simplifies Equation~\ref{eq: Bloch equation rot} and gives the following compact form useful in solving for transverse components of magnetization after RF excitation pulse:
%.........................................................
\begin{equation}\label{eq: Bloch transverse}
	\left(\frac{d\bar{M}}{dt}\right)_\mathrm{rot}=-i\gamma\bar{M}\left(B_0-\frac{\Omega}{\gamma}\right)+i\gamma M_z
B_1(t)e^{-i(\omega_{\mathrm{rf}}-\Omega)t}
\end{equation}
%.........................................................
%=========================================================
\section{Building Blocks of MRI Pulse Sequences}
%=========================================================
An MRI pulse sequence is a set of programmed time-varying magnetic field pulses applied to the object of study placed inside an MRI system. 
There are two categories of pulses: 
\begin{itemize}
\item \textit{RF-pulses}: used for a number of different purposes, primarily to create measurable signal (excitation), form an echo for the signal lost during free induction decay (refocusing), selectively null the signal for certain types of tissues (inversion).
\item \textit{Gradients}: the major purpose is to create linear variation in the external magnetic field that can be exploited to spatially encode MR signal, which is further reconstructed into an image (imaging gradients), control signal phase due to motion (motion-sensitizing) and various correction gradients reducing imaging artifacts~\cite{RNDT24} such as crusher gradients, eddy-current compensation and spoiler gradients. 
\end{itemize}
An infinite number of shapes for RF-pulses and gradients can be used in MRI~\cite{RNDT24, Brown:2014uy} in this section my focus is on those pulses that I used in my experiments.
%~~~~~~~~~~~~~~~~~~~~~~~~~~~~~~~~~~~~~~~~~~~~~~~~~~~~~~~~~
\subsection{RF-pulses}
%~~~~~~~~~~~~~~~~~~~~~~~~~~~~~~~~~~~~~~~~~~~~~~~~~~~~~~~~~
Every MRI pulse sequence must have at least one excitation RF-pulse. 
The purpose of excitation RF-pulse is to flip the magnetization vector $\mathbf{M}$ to allow measurements of MR signal. 
Hence flip angle $\theta$ is the major characteristics of the RF excitation pulses:
%.........................................................
\begin{equation}\label{eq: Flip Angle}
\theta (t) = \gamma \int \limits_{0}^t dt' B_{1}(t')
\end{equation}
%.........................................................
For spin-echo sequence~\cite{Hahn} a flip angle of $\SI{90}{\degree}$ is used while gradient echo sequences utilize a range of values $5-\SI{70}{\degree}$~\cite{RNDT24}.
To calculate magnetization after applying RF-pulse Equation~\ref{eq: Bloch transverse} is used leading to the solution given in~\cite{Joseph:1998fo}:
%.........................................................
\begin{equation}\label{eq: Bloch RF solution}
	\bar{M}\left( t \right) = i \gamma e^{-i\Delta\omega t} \int\limits_{0}^t dt' M_z\left( t' \right) B_1 (t')e^{i\Delta\omega t'}
\end{equation}
%.........................................................
where $\Delta \omega$ is off resonance angular frequency offset. 
This equation states that transverse complex magnetization is directly proportional to inverse Fourier transform of the product of longitudinal component of the magnetization $M_z$ and strength of the magnetic field $B_1$. 
Since $M_z$ is not constant while RF-pulse is applied a small flip angle approximation is usually used ($M_z \approx M_0$)~\cite{RNDT24} resulting in:
%.........................................................
\begin{equation}\label{eq: flip angle dependency}
	\sin{\theta} (\Delta \omega) \approx \theta (\Delta \omega)  \approx \pm \gamma \left| \int\limits_{0}^t dt' B_1 (t')e^{i\Delta\omega t'} \right|
\end{equation}
%.........................................................
Which means that the slice profile can be calculated by performing inverse Fourier transformation of the RF waveform, subject to low flip angle $\theta$. 
Another important characteristics of the RF-pulse is the carrier wave frequency usually chosen at or near Larmor frequency. 
Currently full body human scanners have the $\mathbf{B_0}$ field strength up to $\SI{7}{\tesla}$~\cite{Nowogrodzki:2018eb} giving the RF-pulses frequency range within $1 - \SI{300}{\MHz}$ for $\mathrm{^1H}$ imaging.
%---------------------------------------------------------
\subsubsection{SINC pulses}
%---------------------------------------------------------
One of the most common choice for an excitation RF-pulse is SINC pulse. 
The RF envelope of SINC pulse can be written as following:
%.........................................................
\begin{equation}\label{eq: SINC}
\begin{aligned}
B_1(t) = \begin{cases} A t_0 \dfrac{\sin \left( \tfrac{\pi t}{t_0}\right)}{\pi t},& -N_L t_0 \leq t \leq N_R t_0\\
0,& \mathrm{elsewhere}
\end{cases}
\end{aligned}
\end{equation}
%.........................................................
where $A$ is the amplitude at $t = 0$, and $N_L$ and $N_R$  are the number of zero-crossings in the SINC pulse to the left and fight of the central peak, respectively. 
%~~~~~~~~~~~~~~~~~~~~~~~~~~~~~~~~~~~~~~~~~~~~~~~~~~~~~~~~~
If flip angle is small and the pulse is applied for infinitely long time it will result in a slice profile described by rect($\Delta \omega$) function. 
Infinitely-long pulses are obviously impractical thus SINC pulse is usually truncated and apodized. 
It is primarily used for selective excitation, saturation and refocusing. 
Greatest advantage of SINC pulse is its uniform slice profile.
%--------------------------------------------------------- 
\subsubsection{The Shinnar-Le Roux (SLR) pulse}
%---------------------------------------------------------
SLR pulse design algorithm~\cite{Leroux:1990tk} allows to calculate parameters of RF-pulse from a given slice profile and  the orientation of magnetization vector $\mathbf{M}$. 
The algorithm is based on two key concepts: rotations in three-dimensional space and hard pulse approximation~\cite{RNDT24}. 
In three dimensions rotations are equivalently represented by $\mathbf{SO(3)}$ or $\mathbf{SU(2)}$ groups. 
In $\mathbf{SU(2)}$ representation the rotation matrix $\mathbf{Q}$ is:
%.........................................................
\begin{equation}\label{eq: Rotation matrix Q}
\mathbf{Q} = 
	\begin{bmatrix}
    \alpha & -\beta^*\\
    \beta & \alpha^*\\
   	\end{bmatrix}
\end{equation}
%.........................................................
where $\alpha, \beta$ are Cayley-Klein parameters. 
Matrix $\mathbf{Q}$ contains total of four real numbers which together with three rotation angles needed to describe rotation in three dimensions results into a single constraint obtained from matrix $\mathbf{Q}$ being unitary:
%.........................................................
\begin{equation}\label{eq: C-K normalization}
\alpha \alpha^*+\beta\beta^* = 1
\end{equation}
%.........................................................
The total rotation as a result of multiple pulses is given:
%.........................................................
\begin{equation}\label{eq: Q rotations}
\mathbf{Q} = \mathbf{Q}_{i}\mathbf{Q}_{i-1}\cdots\mathbf{Q}_{1}
\end{equation}
%.........................................................
Hard pulse approximation allows the soft pulse to be approximated by a set of hard pulses separated by free precession time~(Figure~\ref{fig:SLR}).
%*********************************************************
\begin{figure}[!ht]
\vspace{+0.2cm}
\centering
\includegraphics[width=\textwidth]{Figures/SLR.pdf}
\caption[Hard pulse approximation]{Hard pulse approximation.}
\label{fig:SLR}
\end{figure}
%*********************************************************
The effect from a series of hard pulses is approximated by series of two consecutive rotations. 
The first rotation being free precession due to local gradient field by an angle $-\gamma G x \Delta t$ and the second being the rotation about applied RF vector by an angle $-\gamma B_1 \Delta t$~\cite{Pauly:1991ge}. 
Representing the $j$-th state after two consecutive rotations by spinor $s_j$ gives:
%.........................................................
\begin{equation}\label{eq: SLR j-th rotation}
s_j =
\begin{bmatrix}
    \alpha_j\\
    \beta_j\\
\end{bmatrix} = 
	\begin{bmatrix}
    C_j & -S_j^*\\
    S_j & \phantom{-}C_j\\
	\end{bmatrix}
	\begin{bmatrix}
    z^{1/2} & 0\\
    0 & z^{-1/2}\\
	\end{bmatrix}
	\begin{bmatrix}
    \alpha_{j-1}\\
    \beta_{j-1}\\
	\end{bmatrix}
\end{equation}
%.........................................................
where:
%.........................................................
\begin{equation}\label{eq: SLR two rotations}
\begin{aligned}
C_j &= \cos(\gamma \vert B_{1,j} \vert \Delta t /2) \\
S_j &= i e^{i \angle B_{1,j} }\sin(\gamma \vert B_{1,j} \vert \Delta t /2) \\
z &= e^{i \gamma G x \Delta t}
\end{aligned}
\end{equation}
%.........................................................
After N hard pulses are applied N-th state is given:
%.........................................................
\begin{equation}\label{eq: SLR rotation N}
s_N = z^{N/2}
\begin{bmatrix}
    A_N(z)\\
    B_N(z)\\
\end{bmatrix}
\end{equation}
%.........................................................
where $A_N(z)$ and $B_N(z)$ are two polynomials:
%.........................................................
\begin{equation}\label{eq: SLR rotation N}
\begin{aligned}
	A_N(z) & = \sum\limits_{n-1}^{j=0}\alpha_{j}z^{-j}\\[1mm]
	B_N(z) & = \sum\limits_{n-1}^{j=0}\beta_{j}z^{-j}
\end{aligned}
\end{equation}
%.........................................................
these polynomials must also satisfy normalization condition, which follows from Equation~\ref{eq: C-K normalization}:
%.........................................................
\begin{equation}\label{eq: C-K normalization polynom}
|A_N(z)|^2 + |B_N(z)|^2 = 1
\end{equation}
%.........................................................
Thus RF-pulse design becomes the inverse SLR transformation problem, where the profile of RF-pulse is calculated from the set of polynomials Equation~\ref{eq: SLR rotation N} that represent the initial and final state of the magnetization vector $\mathbf{M}$. 
The only missing element at this stage is interpretation of $\mathbf{SU(2)}$ results in terms of $\mathbf{SO(3)}$ since the quantity of interest is magnetization, which is a real three-dimensional vector. 
In $\mathbf{SO(3)}$ representation~\cite{1955PhRevJaynes}, components of the magnetization vector are given by the following equation:
%.........................................................
\begin{equation}\label{eq: PauliSO3_mag}
\begin{bmatrix}
    \bar{M}\phantom{^*}(+)\\
    \bar{M}^{*}(+)\\
    {M}_{z}(+)\\
\end{bmatrix} = 
	\begin{bmatrix}
    (\alpha^*)^2 & -\beta^2 & 2\alpha^*\beta\\
    -(\beta^*)^2 & \alpha^2 & 2\alpha\beta^*\\
    -(\alpha\beta)^* & -\alpha\beta & \alpha\alpha^*-\beta\beta^*\\
	\end{bmatrix}
	\begin{bmatrix}
    \bar{M}\phantom{^*}(-)\\
    \bar{M}^{*}(-)\\
    {M}_{z}(-)\\
    \end{bmatrix}
\end{equation}
%.........................................................
Using Equation~\ref{eq: PauliSO3_mag} the components of magnetization vector for three most interesting special cases of given initial conditions can be summarized in a Table~\ref{tab: SLR-Table}.
%=========================================================
\begin{table}[!htb]
\vspace{+0.2cm}
\caption[Magnetization vector response to Shinnar-Le Roux RF-pulses]{Magnetization vector response to Shinnar-Le Roux RF-pulses.}
\label{tab: SLR-Table}
\begin{center}
\begin{tabular}{@{}lcc@{}}
\toprule[1pt]\midrule[0.3pt]
Pulse Type               & \multicolumn{1}{c}{\begin{tabular}[c]{@{}c@{}}Initial Condition\\ $(M_x, M_y, M_z)$\end{tabular}} & Final State \\ \midrule
Excitation or saturation &     (0, 0, $M_0$)              &    \begin{tabular}[c]{@{}l@{}}$M_x=2M_0\text{Re}(\alpha^*\beta)$\\[1mm] $M_y=2M_0\text{Im}(\alpha^*\beta)$\\[1mm] $M_z=2M_0\alpha^*\beta$\end{tabular}         \\[9mm]
Inversion                &      (0, 0, $M_0$)             &     $M_z=M_0(\alpha\alpha^*-\beta\beta^*)$   \\[3mm]
Refocus &       ($\bar{M}$, $\bar{M^*}$, 0)            &        $\bar{M}=\bar{M}(\alpha^*)^2-\bar{M^*}\beta^2$     \\[1mm] \midrule[0.3pt]\toprule[1pt]
\end{tabular}
\end{center}
\vspace{-0.2cm}
\end{table}
%=========================================================
Although SLR design algorithm accounts for nonlinearity of Bloch equations the procedure should be repeated every time that the flip angle is changed. 
Among the advantages is the ability to make trade-offs between the parameters of RF-pulse, such as pulse duration, bandwidth, passband and stopband ripple, flip angle.
%---------------------------------------------------------
\subsubsection{Spectral Spatial Excitation Pulses (SPSP)}
%---------------------------------------------------------
This type of RF-pulse excites magnetization at specified location and with the specified spectral content~\cite{Schick:1998hu, Block:1997cv}. 
Among the advantages of SPSP pulses is the ability to replace two different conventional pulses as well as high tolerance to the $B_1$ field inhomogeneities. 
Signal from the SPSP pulses consist of a multiple short RF sub-pulses modulated by broad RF envelope combined with the bipolar slice-selective gradients~Figure~\ref{fig:SPSP}. 
The broad RF envelope performs spectral selection while sub-pulses and slice selective gradients are responsible for spatial selection.
%*********************************************************
\begin{figure}[!ht]
\vspace{+0.2cm}
\centering
\includegraphics[scale=.35]{Figures/SPSP.pdf}
\caption[Spectral Spatial RF-pulse]{Spectral Spatial RF-pulse (SPSP).}
\label{fig:SPSP}
\end{figure}
%*********************************************************
When designing RF-pulse it is convenient to use the concept of RF \mbox{\textit{k-}space}. 
Defining $k_{\omega}$ and $k_{z}$ to be the spectral and spatial axis respectively, then:
%.........................................................
\begin{equation}\label{eq: SPSP_kspace}
\begin{aligned}
	k_z &= \frac{\gamma}{2 \pi}\int\limits_{T_{\mathrm{end}}}^t dt' \, G_{sl}(t')\\
	k_{\omega} &= T_{\mathrm{end}}-t
\end{aligned}
\end{equation}
%.........................................................	
where $G_{sl}$ is a slice selective gradient, and $T_{\mathrm{end}}$ is the end of slice rephasing gradient. 
The design of SPSP pulses includes several important stages: choice of of the slice-selection gradient form as well as it's amplitude and frequency, the placement of the desired and undesired components relative to the frequency side-lobes, the form of the RF envelope, which determines the \mbox{\textit{k-}space} weighting and thus the spatial and spectral slice profiles, the length of the pulse and the modulation of these pulses to shift them in space and frequency~\cite{RNDT24}. 
	\begin{itemize}
	\item For the slice-selection gradient oscillating trapezoidal lobes with minimal rise time and maximum amplitude are the possible choice. The oscillations are needed to sample $k_{\omega}$ axis. Such selection minimizes slice thickness since $\Delta z \propto 1/k_{z}$. 
	\item The desired and undesired components in human MRI are usually water and fat signals respectively. Simplest option is to place water at the central lobe and fat at the null between the main lobe and the first side-lobe, so called $\mathit{true-null}$ design~\cite{Meyer:1990cv}. For example a system with $B_0 = \SI{1.5}{\tesla}$ has $f = \SI{220}{\hertz}$. The required gradient modulation frequency for this method is twice the water/fat frequency difference. The period for the slice-selective gradient is then:
%.........................................................
\begin{equation}\label{eq: SPSP_period}
	T=\frac{1}{2f}=\SI{2.27}{\milli\second}
\end{equation}
%.........................................................	
The width for each trapezoidal lobe is then $T/2 = \SI{1.14}{\milli\second}$. 
Alternatively for the systems with low performance gradients hardware $\mathit{opposed-null}$ method could be used~\cite{RNDT24}. 
The period of the oscillating gradients is chosen to correspond to the closest secondary peak, increasing the width for each trapezoidal lobe by the factor of two. 
This method is however prone to partial volume artifacts, thus is less desirable to use.
\item Lastly for the choice of the RF envelope three factors are taken into account~\cite{RNDT24}.
%.........................................................
\begin{equation}\label{eq: SPSP_RF}
B_1(t)=C(\theta)|G_z(t)|A_{spec}(t)A_{spat}(k_z)
\end{equation}
where $C(\theta)$ is a normalization factor:
%.........................................................
\begin{equation}\label{eq: SPSP_C}
	C(\theta)=\frac{1}{\gamma\displaystyle\int dt A_{spec}(t)A_{spat}(k_z)}
\end{equation}
%.........................................................	
and $A_{spec}$, $A_{spat}$ is a spectral modulation envelope and spatial kernel. 
Possible choices for both $A_{spec}$ and $A_{spat}$ are SINC and SLR pulses.
\end{itemize}
A well known disadvantage of SPSP pulses is that due to the short duration of the individual sub-pulse RF they are not very spatially selective and as a result have small time-bandwidth products. 
That leads to either  broad spatial profiles transition regions or large minimum-slice thickness. 
To address these deficiencies, SPSP pulses are typically played with a high RF duty cycle~\cite{RNDT24}. 
Another issue with SPSP pulses is related to the fact that RF sub-pulses are played simultaneously with rapidly oscillating slice-selective gradients. 
If eddy currents of slice-selective gradients are not accurately compensated, the actual \mbox{\textit{k-}space} trajectory will deviate from expected and the performance of the SPSP pulse will degrade.
%~~~~~~~~~~~~~~~~~~~~~~~~~~~~~~~~~~~~~~~~~~~~~~~~~~~~~~~~~
\subsection{Gradients}
%~~~~~~~~~~~~~~~~~~~~~~~~~~~~~~~~~~~~~~~~~~~~~~~~~~~~~~~~~
While excitation RF-pulses create measurable MR signal, gradients perform its spatial and motion encoding. 
Gradients are created by pairs of coils capable of producing additional spatial linear variation to the external field $\mathbf{B}$ in each of the three orthogonal directions. 
Considering a simple case where gradient along the $x$ axis is introduced, the  magnetic field is then expressed as:
%.........................................................
\begin{equation}\label{eq: Field with gradient}
	\mathbf{B} = \mathbf{B_0} + G_x(t) \mathbf{x}
\end{equation}
%.........................................................	
As a result of the spatial linear variation of the magnetic field the precession frequency for the protons will also vary linearly according to Equations~\ref{eq: Quantum Larmor frequency}~and~\ref{eq: Field with gradient}.
%*********************************************************
\begin{figure}[!h]
\vspace{+0.2cm}
\centering
\includegraphics[scale=.4]{Figures/ProtonGradient.pdf}
\caption[Precession of protons in the magnetic field with the spatial linear gradient]{Precession of protons in the magnetic field with the spatial linear gradient applied.}
\label{fig: ProtonGradient}
\end{figure}
%*********************************************************
 Protons subject to stronger magnetic field $\mathbf{B}$ precess faster, while those subject to weaker field precess slower~(Figure~\ref{fig: ProtonGradient}). 
 More general definition of gradients can be given in form of Jacobian:
%.........................................................
\begin{equation}\label{eq: Gradients}
	\mathbf{G}(t) = 
	\begin{bmatrix}
    \dfrac{\partial B_x(t)}{\partial x} & \dfrac{\partial B_x(t)}{\partial y} & \dfrac{\partial B_x(t)}{\partial z}\\[8pt]
    \dfrac{\partial B_y(t)}{\partial x} & \dfrac{\partial B_y(t)}{\partial y} & \dfrac{\partial B_y(t)}{\partial z}\\[8pt]
    \dfrac{\partial B_z(t)}{\partial x} & \dfrac{\partial B_z(t)}{\partial y} & \dfrac{\partial B_z(t)}{\partial z}\\
	\end{bmatrix}
\end{equation}
%.........................................................
For most applications the desired shape of the gradient pulse is $\mathrm{rect}(t)$ function, which would allow instantaneous change in the magnetic field. 
Yet in practice rectangular gradient pulses are impossible and are commonly approximated by trapezoid. 
Two important characteristics of the gradient pulses limited by hardware capabilities are amplitude $G_{\mathrm{max}}$ and slew rate $\sigma = G_{\mathrm{max}} / \epsilon_{\mathrm{rise}}$ ~(Figure~\ref{fig: TrapGrad}).
%*********************************************************
\begin{figure}[!htb]
\vspace{+0.2cm}
\centering
\includegraphics[scale=.48]{Figures/Trapezoid.pdf}
\caption[Trapezoid gradient shape]{Trapezoid gradient shape.}
\label{fig: TrapGrad}
\end{figure}
%*********************************************************
For modern MRI systems gradients can be applied with slew rate up to $\SI{500}{\tesla / \meter / \second}$ and amplitude up to $\SI{70}{\milli\tesla / \meter}$~\cite{Tan:2020ht}, still the full performance sometimes may not be used to avoid peripheral nerve simulation~\cite{Ham:1997is}. 
%---------------------------------------------------------
\subsubsection{Imaging gradients}
%---------------------------------------------------------
Every MRI pulse sequence incorporates imaging gradients. 
Signal is encoded spatially in 3D using three types of gradients:
\begin{itemize}
	\item \textit{Slice selection}: First imaging gradient in the pulse sequence which accompanies the initial RF excitation pulse. Slice selection gradient is applied perpendicularly to the desired slice plane and translates the band of frequencies into the desired band of locations. For the same RF-pulse band stronger slice selection gradient results in thiner slice.
	\item \textit{Phase-encoding}: Once magnetization in the selected slice is tipped to the transverse plane phase-encoding gradient is applied. Same as shown in Figure~\ref{fig: ProtonGradient} protons will experience change in angular frequency. 
%*********************************************************
\begin{figure}[!htb]
\vspace{+0.2cm}
\centering
\includegraphics[scale=.45]{Figures/PhaseEncode.pdf}
\caption[Accumulated phase due to applied gradient]{Accumulated phase due to applied gradient.}
\label{fig: PhaseEncode}
\end{figure}
%********************************************************* 
	The phase-encoding gradient is then switched~off eliminating frequency variation but keeping the variation in accumulated phase intact~ Figure~\ref{fig: PhaseEncode}. Procedure should be repeated to collect data at multiple phase-encoded levels. This is achieved by varying the area under phase-encoding gradient.
	\item \textit{Frequency-encoding}: Applied continuously while the MR signal is measured it introduces spatial variation in angular frequency.\end{itemize}
Together phase- and frequency-encoding introduce an import MRI concept -- \textbf{\mbox{\textit{k-}space}}, which can be defined as a data matrix that holds phase- and frequency-encoded MR signal related to image matrix by Fourier transform.
%---------------------------------------------------------
\subsubsection{Motion-Sensitizing}
%---------------------------------------------------------
First described by Stejskal and Tanner~\cite{Stejskal} and commonly placed in the MRI sequence as a set of balanced bipolar gradients. 
These gradients are utilized to quantify the dynamics of  tissue or fluid such as diffusion and flow. 
Application of motion-sensitizing gradients for Velocity Encoded Phase-Contrast (VEPC) imaging and Diffusion Weighted Imaging (DWI) is discussed in more details in Chapters~\ref{ch: VEPC}~and~\ref{ch: Diffusion} respectively.
%---------------------------------------------------------
\subsubsection{Spoiler and Crusher}
%---------------------------------------------------------
Auxiliary gradients used to manipulate phase coherence of the magnetization vector. 
Spoiler gradients eliminate transverse magnetization by dephasing, while crushers are used to retain desired signal pathway by either dephasing or rephasing~\cite{RNDT24}. 
Crushers gradients must be used for Stimulated Echo Acquisition Mode (STEAM) which is discussed in more details in~Chapter~\ref{ch: Diffusion}.